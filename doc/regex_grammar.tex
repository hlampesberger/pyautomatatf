\documentclass[a4paper, 12pt]{article}

\usepackage[utf8]{inputenc}
\usepackage[T1]{fontenc}
\usepackage[ngerman]{babel}
\usepackage{a4wide}

\usepackage{textcomp}       % for symbols
\usepackage{microtype}
\usepackage{mathptmx}       % selects Times Roman as basic font
%\usepackage{helvet}         % selects Helvetica as sans-serif font
%\usepackage{courier}        % selects Courier as typewriter font
%usepackage{ascii}
\renewcommand*\ttdefault{txtt}
\usepackage{xcolor}
\usepackage{setspace}
\onehalfspacing


\usepackage{amsmath}
\usepackage{amsfonts}
\usepackage{amssymb}
\usepackage{multirow}
\usepackage{amsthm}

\newtheoremstyle{mystyle}% <name>
{3pt}% <Space above>
{3pt}% <Space below>
{}% <Body font>
{}% <Indent amount>
{\bf}% <Theorem head font>
{:}% <Punctuation after theorem head>
{.5em}% <Space after theorem headi>
{}% <Theorem head spec (can be left empty, meaning `normal')>
\theoremstyle{mystyle}
\newtheorem{definition}{Definition}
\newtheorem{task}{Aufgabe}
%\newtheorem{task
\newtheorem{example}{Beispiel}

\usepackage[hyphens]{url}
\usepackage{hyperref}
\hypersetup
{
    pdfauthor={Harald Lampesberger},
    pdftitle={Intrusion Detection},
    citecolor=blue,
    urlcolor=black,
    linkcolor=black,
    colorlinks=true
}

\definecolor{blue}{rgb}{0,0,0.75}
\definecolor{green}{rgb}{0,0.75,0}
\definecolor{lgray}{rgb}{0.9,0.9,0.9}
\definecolor{deepblue}{rgb}{0,0,0.5}
\definecolor{deepred}{rgb}{0.6,0,0}
\definecolor{deepgreen}{rgb}{0,0.5,0}

\usepackage{listings}
\lstset{
language=Python,
basicstyle=\ttfamily\normalsize,
otherkeywords={self, bytes, float},
keywordstyle=\color{blue}\bfseries,
emph={},          % Custom highlighting
emphstyle=\ttb\color{deepred},    % Custom highlighting style
stringstyle=\color{deepgreen},
%frame=tb,                         % Any extra options here
showstringspaces=false            % 
}

\usepackage{tikz}
\usepackage{tikz-qtree}
\usetikzlibrary{shapes,arrows,trees,positioning}
\usetikzlibrary{decorations.markings,decorations.pathmorphing}

\usepackage{dirtree}

\title{Multi Regex Matching}
\author{Sichere Informationssysteme, Intrusion Detection Systems ILV\\
\normalsize Harald Lampesberger}
\date{\normalsize Sommersemester 2016}
\begin{document}
%\maketitle




\begin{eqnarray}
\mathbf{rootExpression} & \to & \text{\color{red}\texttt{\^}}? \; expression \; \text{\color{red}\texttt{\$}}? \\
expression & \to & choice \mid term \mid \epsilon \\
choice & \to & term \; \text{\color{red}\texttt{|}} \; expression \\
term & \to & sequence \mid factor \\
sequence & \to & factor \; term \\
factor & \to & iteration \mid eventually \mid atom \\
atom & \to & anyCharacter \mid characterClass \mid character \mid \text{\color{red}\texttt{(}} \; expression \; \text{\color{red}\texttt{)}} \\
iteration & \to & atom \; \text{\color{red}\texttt{*}} \\
%oneOrMore & \to & atom \; \text{\color{red}\texttt{+}} \\
eventually & \to & atom \; \text{\color{red}\texttt{?}} \\
%quantified & \to & atom \; \text{\color{red}\texttt{\{}} \; min \mid (min \; \text{\color{red}\texttt{,}}) \mid (min \; \text{\color{red}\texttt{,}} \; max) \mid (\text{\color{red}\texttt{,}} \; max) \; \text{\color{red}\texttt{\}}} \\
%min = max & \to & digit^+ \\
%characterClass & \to & \text{\color{red}\texttt{[}} \; \text{\color{red}\texttt{\^}}? \; (character \mid character \; \text{\color{red}\texttt{-}} \; character)^+ \text{\color{red}\texttt{]}} \\
characterClass & \to & \text{\color{red}\texttt{[}} \; (escapedCharacter \mid alphanum \mid meta)^+ \text{\color{red}\texttt{]}} \\
character & \to & escapedCharacter \mid alphanum \mid meta \\
escapedCharacter & \to & \text{\color{red}\texttt{\textbackslash\textbackslash}} \mid \text{\color{red}\texttt{\textbackslash x}} \; hex \; hex \mid  \; \text{\color{red}\texttt{\textbackslash}} \; esc \\
anyCharacter & = & \text{\color{red}\texttt{.}} \\
%digit & \in & \text{\color{red}\texttt{0123456789}} \\
alphanum & \in & \text{\color{red}\texttt{abcde}} \dots \text{\color{red}\texttt{zABCDE}} \dots \text{\color{red}\texttt{Z01}} \dots \text{\color{red}\texttt{789}} \\
hex & \in & \text{\color{red}\texttt{0123456789ABCDEFabcdef}} \\
esc & \in & \text{\color{red}\texttt{(){}[]*|+\textasciicircum \$/.?tnrf-}} \\
meta & \in & \text{\color{red}\texttt{\textvisiblespace \#!"\%§\&'/,:;<=>@\_-}$\sim$}
\end{eqnarray}

%\vspace{1em}

%\begin{table}
%\centering
%\label{tab:operators}
%\caption{Unterstützte Operatoren für unsere eingeschränkte Regex Syntax.}
\begin{tabular}{l|l}
Operator                                                              & Beschreibung                            \\ \hline
\color{red}\texttt{a} & Einzelnes ASCII Zeichen (Byte) \\
\color{red}\texttt{\textbackslash}\color{black}$s$ & Escape-Sequenz für Sonderzeichen (Byte), z.B. Klammern   \\
\color{red}\texttt{\textbackslash x}\color{black}$h_1h_2$ & Hex-Code für ein Zeichen (Byte)   \\
\color{red}\texttt{.}                                             & Wildcard, irgendein Zeichen (Byte) \\
\color{red}\texttt{[AaBbCc]}                                     & Alternativen für ein Zeichen (Byte), keine Ranges \\
\color{red}\texttt{abracadabra} & Sequenz \\
$expr_1$\color{red}\texttt{|}\color{black}$expr_2$                                             & Alternative (Infix) \\
$expr$\color{red}\texttt{*}                   & $0\dots\infty$ Wiederholung (links-assoziativ)            \\
%$expr$\color{red}\texttt{+}                   & $1\dots\infty$ Wiederholung (links-assoziativ)            \\
%$expr$\color{red}\texttt{\{}{\color{black}$x$}\texttt{\}}                                       & Wiederholung exakt $x$-mal (links-assoziativ)         \\
$expr$\color{red}\texttt{?}                                             & Optional (links-assoziativ)             \\
{\color{red}\texttt{(}}$expr${\color{red}\texttt{)}}                                           &   Klammern für Gruppierung (Reihenfolge)   \\
{\color{red}\texttt{\^}}$expr${\color{red}\texttt{\$}} & linker und rechter Anker (beide optional) \\
\end{tabular}
%\end{table}



\end{document}